\section{Fundamentação teórica}

\subsection{Avaliação de desempenho de sistemas}
\begin{frame}{Avaliação de desempenho de sistemas}
    \begin{itemize}
        \item A
        \item B
        \item C 
        \item D 
    \end{itemize}
\end{frame}

\subsection{Green software}
\begin{frame}{Green software}
    \begin{itemize}
        \item Desenvolvimento e uso de software de forma sustentável

    \end{itemize}
\end{frame}

\subsection{Eficiência energética}
\begin{frame}{Eficiência energética}
    \begin{itemize}
        \item A
        \item B
        \item C 
        \item D 
    \end{itemize}
\end{frame}

\subsection{Métricas energéticas}

\begin{frame}{Joule}
    \begin{itemize}
        \item É a unidade de energia no Sistema Internacional de Unidades, utilizada para medir energia mecânica ou térmica
        \item Na energia mecânica 1 Joule equivale a energia nescessária aplicar força 1 Newton por 1 metro
        \item Na energia termica 1 Joule equivale a energia nescessária para aumentar a temperatuda da água a 1 grau
    \end{itemize}

\end{frame}

\begin{frame}{Watt}
    \begin{itemize}
        \item É a unidade de potencia no Sistema Internacional de Unidades, sendo a potencia media da quantidade de energia em um determinado tempo
        \item 1 Watt equivale a 1 Joule por segundo, logo, um dispositivo que consome 1 Watt está consummindo um Joule por segundo
    \end{itemize}
    \begin{equation}
        P = \frac{E}{t}
    \end{equation}
\end{frame}

\begin{frame}{Quilowatt}
    \begin{itemize}
        \item É uma unidade de potência referente a 1000 Watts
        \item Utilizada para medir portencia eletrica em aplicações residenciais e comerciais e industriais
    \end{itemize}
    \begin{equation}
        \text{Potência em quilowatts (kW)} = \frac{\text{Energia em kilojoules (kJ)}}{\text{Tempo em horas (h)}}
        \end{equation}
\end{frame}

\begin{frame}{Quilowatt-hora}
    \begin{itemize}
        \item Referente a energia produzida ou consumida no período de 1 hora
        \begin{equation}
            \text{Energia total (kWh)} = \text{Potência em Quilowatts (kW)} \times \text{Tempo total em horas (h)}
            \end{equation}
    \end{itemize}
\end{frame}


