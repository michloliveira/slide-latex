\section{Fundamentação teórica}

% \subsection{Avaliação de desempenho de sistemas}
% \begin{frame}{Avaliação de desempenho de sistemas}
%     \begin{itemize}
%         \item Objetivo de entender como um sistema se comporta em termos de um determinada métrica com uma carga de trabalho;
%         \item Auxiliar na tomada de decisão como investimentos em hardware e ou software;
%         \item Otimização e melhorias a partir da análise dos resultados.
%     \end{itemize}
% \end{frame}

% \subsection{Green software}
% \begin{frame}{Green software}
%     \begin{itemize}
%         \item Abordagem para o desenvolvimento e uso do software de maneira mais sustentável;
%         \item Objetivo de reduzir impactos ambientais;
%         %\item Estratégias que buscam minimizar o consumo de recursos naturais.

%     \end{itemize}
% \end{frame}

\subsection{Eficiência energética}
\begin{frame}{Eficiência energética}
    \begin{itemize}
        \item Utilização da energia de forma consciente e eficaz.
        \item Objetivo de minimizar desperdícios
        \item Otimizar o uso de energia de forma que a energia consumida seja proporcional ao trabalho realizado.
    \end{itemize}
\end{frame}

\subsection{Métricas energéticas}

\begin{frame}{Joule}
    \begin{itemize}
        \item É a unidade de energia no SI, utilizada para medir energia mecânica ou térmica;
        \item Na energia mecânica, 1 Joule equivale a energia necessária para aplicar força 1 Newton por 1 metro;
        \item Na energia térmica, 1 Joule equivale a energia necessária para aumentar a temperatura da água a 1 grau.
    \end{itemize}

\end{frame}

\begin{frame}{Watt}
    \begin{itemize}
        \item É a unidade de potência no SI;
        \item 1 Watt equivale a 1 Joule por segundo, logo, um dispositivo que consome 1 Watt está consumindo um Joule por segundo.
    \end{itemize}
    \begin{equation}
        P = \frac{E}{t}
    \end{equation}
\end{frame}

\begin{frame}{Quilowatt}
    \begin{itemize}
        \item É uma unidade de potência referente a 1000 Watts;
        %\item Utilizada para medir potência elétrica em aplicações residenciais, comerciais e industriais;
    \end{itemize}
    \begin{equation}
        \text{Potência em quilowatts (kW)} = \frac{\text{Energia em kilojoules (kJ)}}{\text{Tempo em horas (h)}}
        \end{equation}
\end{frame}

\begin{frame}{Quilowatt-hora}
    \begin{itemize}
        \item Referente a energia produzida ou consumida no período de 1 hora.
        \begin{equation}
            \text{Energia total (kWh)} = \text{Potência em Quilowatts (kW)} \times \text{Tempo total em horas (h)}
            \end{equation}
    \end{itemize}
\end{frame}


